%%%%%%%%%%%%%%%%%%%%%%%%%%%%%%%%%%%%%%%%%%%%%%%%%%%%%%%%%%%%
%                                                           %
% FH St. Pöten                                              %
%                                                           %
% LaTeX-Layout für Protokolle                               %
%                                                           %
% (c) 2013 Tobias Millauer http://www.tobias-millauer.com/  %
%                                                           %
%%%%%%%%%%%%%%%%%%%%%%%%%%%%%%%%%%%%%%%%%%%%%%%%%%%%%%%%%%%%%

\documentclass{scrartcl}
\usepackage[utf8]{inputenc}
\usepackage[ngerman]{babel}
\usepackage{hyphenat}

\usepackage{graphicx}
\usepackage{booktabs}
\usepackage{multirow}
\usepackage{array}
\usepackage{tabularx}
\usepackage{fancyhdr}
\usepackage{titlesec}
\usepackage[usenames,dvipsnames]{color}
\usepackage[left=2cm,right=2cm,top=2cm,bottom=2cm,includeheadfoot]{geometry}

% TODO: https://asciich.ch/wordpress/seitenrander-in-latex/

%%%%%%%%%%%%%%%%%%%%%%%%%%%%%%%%%%%%%%%%%%%%%%%%%%%%%%%%
% User Variables

\newcommand{\fhUeNummer}{1}
\newcommand{\fhBeschreibung}{Uebungstitel}
\newcommand{\fhStudium}{IT Security}
\newcommand{\fhStudiengang}{BIS-VZ}
\newcommand{\fhFach}{ENVS}
\newcommand{\fhVortragender}{\begin{Large}Vortragender\end{Large}}

\newcommand{\fhTitel}{UE~\fhUeNummer~-~\fhBeschreibung}
\newcommand{\fhAutoren}{}
\newcommand{\fhUntertitel}{\fhFach~\fhStudiengang}

%%%%%%%%%%%%%%%%%%%%%%%%%%%%%%%%%%%%%%%%%%%%%%%%%%%%%%%%

\parindent0pt
\setlength{\parskip}{10pt}
\renewcommand{\familydefault}{\sfdefault}
\newcolumntype{C}[1]{>{\centering\arraybackslash}m{#1}}
\definecolor{fhColor}{RGB}{63, 149, 182}
\renewcommand{\labelitemi}{\color{fhColor}{•}}
\renewcaptionname{ngerman}{\figurename}{Abb.}

\titleformat{\section}[hang]{\LARGE\bfseries\color{fhColor}}{}{0pt}{}
\titleformat{\subsection}[hang]{\Large\bfseries}{}{0pt}{}
\titleformat{\subsubsection}[hang]{\large\bfseries}{}{0pt}{}

\pagestyle{fancy}

% Reset
\fancyhf{}
\renewcommand{\headrulewidth}{0.0pt}

% Erste Seite im Kapitel ohne K.u.F.
\fancypagestyle{plain}{
   \fancyhf{} 
}

\setlength{\topskip}{1.5cm}
\setlength{\headsep}{1.5cm}

\fancyhead[L]{
\begin{huge}\color{fhColor}{\fhStudium}\end{huge}\\
\vspace{0.10cm}
\begin{Large}\color{fhColor}{\fhUntertitel}\end{Large}
\begin{flushleft}
\end{flushleft}
\begin{flushright}
\vspace*{-2.5cm}
    \includegraphics[width=0.15\textwidth]{fh_logo_heading} 
\end{flushright}
}

% WHY DOES THIS SHIT NOT WORK?!?
\fancyfoot[C]{
\tiny{Fachhochschule St. Pölten GmbH, Matthias Corvinus-Straße 15, 3100 St. Pölten, T: +43 (2742) 313 228, F: +43 (2742) 313 228-339, E: office@fhstp.ac.at, I: www.fhstp.ac.at}
}

\title{\fhtitel}
\subtitle{\fhStudienbeschreibung}
\author{\fhautoren}
\date{\today}

\begin{document}

\begin{titlepage}

\newgeometry{left=35mm, right=23mm, top=6mm, bottom=15mm}

\begin{figure}[h]
    \hspace{9.1cm}
    \includegraphics[width=0.5\textwidth]{fh_logo}   
\end{figure}

\vspace{-4.2cm}
\begin{huge}{\color{fhColor} IT Security}\end{huge}
\vspace{6cm}

\begin{flushright}

\textbf{\Huge \color{fhColor}{\fhTitel}}\\

\vspace{0.5cm}

\begin{huge} \color{fhColor}{\fhUntertitel}\end{huge}\\

\vspace{4cm}

\begin{tabular}{lllll}
\multirow{2}{*}{\fhVortragender} & \multirow{2}{*}{\begin{Large}/\end{Large}} & \begin{LARGE}Student 1\end{LARGE} & \small{(is1310xx)}\\
& & \begin{LARGE}Student 2\end{LARGE} & {(is1310xx)}
\end{tabular}

\vspace{2cm}

\begin{Large} Wintersemester 2013/2014\end{Large}

\vspace{5cm}

\today
\end{flushright}
\end{titlepage}

\tableofcontents

\section{Überschrift}

Lorem ipsum dolor sit amet, consetetur sadipscing elitr, sed diam nonumy eirmod tempor invidunt ut labore et dolore magna aliquyam erat, sed diam voluptua. At vero eos et accusam et justo duo dolores et ea rebum. Stet clita kasd gubergren, no sea takimata sanctus est Lorem ipsum dolor sit amet. Lorem ipsum dolor sit amet, consetetur sadipscing elitr, sed diam nonumy eirmod tempor invidunt ut labore et dolore magna aliquyam erat, sed diam voluptua.

\subsection{Zwischenüberschrift}

\begin{figure}[h!]
\centering
\includegraphics[width=0.25\textwidth]{Tux}
\caption{Tux}
\label{threadsVsSync}
\end{figure}

\subsubsection{Unterpunkt}

\begin{itemize}
    \item Aufzählung 1
    \item Aufzählung 2
    \item Aufzählung 3
\end{itemize}

\end{document}